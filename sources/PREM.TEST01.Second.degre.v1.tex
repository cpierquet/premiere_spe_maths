% !TeX TXS-program:compile = txs:///arara
% arara: lualatex: {shell: no, synctex: yes, interaction: batchmode}
% arara: lualatex: {shell: no, synctex: yes, interaction: batchmode} if found('log', 'undefined references')

\documentclass[a4paper,11pt]{article}
\usepackage[]{cp-base}
\graphicspath{{./graphics/}}
%variables
\donnees[%
	classe=1\up{ère} 2M2,
	matiere={[SPÉ.MATHS]},
	typedoc=TEST~,
	numdoc=1,
	mois=Mardi 21 Septembre,
	titre={}
]

%formatage
\author{Pierquet}
\title{\nomfichier}
\hypersetup{pdfauthor={Pierquet},pdftitle={\nomfichier},allbordercolors=white,pdfborder=0 0 0,pdfstartview=FitH}
%fancy
\fancypagestyle{entetesujetA}{\fancyhead[R]{\entete{\classe{}A - \mois{} \annee}}}
\fancypagestyle{entetesujetB}{\fancyhead[R]{\entete{\classe{}B - \mois{} \annee}}}
\lhead{\entete{\matiere}}
\chead{\entete{\lycee}}
\rhead{\entete{\classe{} - \mois{} \annee}}
\lfoot{\pied{\matiere}}
\cfoot{\logolycee{}}
%\rfoot{\pied{\numeropagetot}}

\begin{document}

\pagestyle{fancy}

\part{TEST01 - Second degré}

\medskip

\nomprenomtcbox

\medskip

\begin{blocexo}Questions de cours\end{blocexo} %exo1

\medskip

Soit $ax^2+bx+c$ un trinôme, avec $a \neq 0$.

\begin{enumerate}
	\item Rappeler l'écriture de la forme canonique du trinôme.
\end{enumerate}

\papierseyes{20}{3}

\begin{enumerate}[resume]
	\item Donner la formule permettant de calculer le discriminant $\Delta$, ainsi que les formules des éventuelles racines.
\end{enumerate}

\papierseyes{20}{5}

\medskip

\begin{blocexo}Exercice\end{blocexo} %exo2

\begin{enumerate}
	\item Mettre le trinôme $f(x)=2x^2-12x-14$ sous forme canonique (on détaillera un minimum les calculs), puis dresser son tableau de variations.
\end{enumerate}

\papierseyes{20}{6}

\begin{enumerate}[resume]
	\item Déterminer les éventuelles racines et l'éventuelle forme factorisée du trinôme $-2x^2+10x+12$.
\end{enumerate}

\papierseyes{20}{6}

\end{document}