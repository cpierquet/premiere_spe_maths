% !TeX TXS-program:compile = txs:///pythonlualatex

\documentclass[a4paper,11pt]{article}
\usepackage[revgoku]{cp-base}
\graphicspath{{./graphics/}}
%variables
\donnees[typedoc=CHAP,numdoc=09,classe=1\up{ère} 2M2,matiere={SPÉ MATHS},mois=Mars,annee=2022]
%formatage
\author{Pierquet}
\title{\nomfichier}
\hypersetup{pdfauthor={Pierquet},pdftitle={\nomfichier},allbordercolors=white,pdfborder=0 0 0,pdfstartview=FitH}
%divers
\lhead{\entete{\matiere}}
\chead{\entete{\lycee}}
\rhead{\entete{\classe{} - \mois{} \annee}}
\lfoot{\pied{\matiere}}
\cfoot{\logolycee{}}
\rfoot{\pied{\numeropagetot}}
\urlstyle{same}

\begin{document}

\pagestyle{fancy}

\part{CH09 - Suites arithmétiques, géométriques - Exercices}

\medskip

\begin{caide}
{\setlength\arrayrulewidth{1.5pt} \arrayrulecolor{titrebleu!35}
\begin{tabularx}{\linewidth}{Y|Y|Y|Y|Y|Y}
	\niveaudif{0}~~\textsf{Basique} & \niveaudif{1}~~\textsf{Modérée} & \niveaudif{2}~~\textsf{Élevée} & \niveaudif{3}~~\textsf{Très élevée} & \niveaudif{4}~~\textsf{Extrême} & \niveaudif{5}~~\textsf{Insensée} \\
\end{tabularx}}
\end{caide}

\exonum{0}

\begin{enumerate}
	\item Soit $\suiten$ la suite arithmétique de 1\up{er} terme $u_0=5$ et de raison $r=6$.
	\begin{enumerate}
		\item Donner la formule de récurrence liée à la suite $\suiten$.
		\item Déterminer la formule explicite donnant $u_n$ en fonction de $n$.
		\item En déduire les valeurs de $u_7$ et de $u_{50}$.
		\item Déterminer, en justifiant, le sens de variations de la suite $\suiten$.
		\item Calculer la somme $u_0 + u_1 + \ldots + u_{20}$.
	\end{enumerate}
	\item Soit $\suiten[v]$ la suite géométrique de 1\up{er} terme $v_1=1000$ et de raison $q=1,05$.
	\begin{enumerate}
		\item Donner la formule de récurrence liée à la suite $\suiten[v]$.
		\item Déterminer la formule explicite donnant $v_n$ en fonction de $n$.
		\item En déduire, arrondies au millième, les valeurs de $v_7$ et de $v_{20}$.
		\item Déterminer, en justifiant, le sens de variations de la suite $\suiten[v]$.
		\item Calculer, arrondie au millième,  la somme $v_1 + v_2 + \ldots + v_{13}$.
	\end{enumerate} 
\end{enumerate}

\smallskip

\exonum{1}

\begin{enumerate}
	\item Soit $\suiten$ la suite définie par $\begin{dcases} u_1 = 50 \\ u_{n+1}=u_n-3 \text{ pour tout entier }n \pg 1 \end{dcases}$.
	\begin{enumerate}
		\item Déterminer la nature de la suite $\suiten$.
		\item Déterminer la formule explicite donnant $u_n$ en fonction de $n$.
		\item Déterminer, en justifiant, le sens de variations de la suite $\suiten$.
		\item Déterminer, par calculs, le plus petit entier naturel $n$ pour lequel $u_n < 0$.
	\end{enumerate}
	\item Soit $\suiten[v]$ la suite définie par $\begin{dcases} v_0 = 100 \\ v_{n+1}=0,88v_n \text{ pour tout entier }n \end{dcases}$.
	\begin{enumerate}
		\item Déterminer la nature de la suite $\suiten[v]$.
		\item Déterminer la formule explicite donnant $v_n$ en fonction de $n$.
		\item Déterminer, en justifiant, le sens de variations de la suite $\suiten[v]$.
		\item Déterminer, en détaillant la méthode, le plus petit entier naturel $n$ pour lequel $v_n \pp 45$.
	\end{enumerate} 
\end{enumerate}

\smallskip

\exonum{2}

\begin{enumerate}
	\item Soit $\suiten$ une suite arithmétique telle que $u_2=10$ et $u_5=46$.
	\begin{enumerate}
		\item Déterminer, en justifiant, la raison $r$ de $\suiten$.
		\item En déduire la valeur de $u_{30}$.
	\end{enumerate}
	\item Soit $\suiten[v]$ une suite géométrique telle que $v_5 = 4$ et $v_7=6,25$.
	\begin{enumerate}
		\item Déterminer, en justifiant, une valeur possible pour la raison $q$ de $\suiten[v]$.
		\item En déduire la valeur de $v_{11}$.
	\end{enumerate}
\end{enumerate}

\pagebreak

\exonum{1}

\begin{enumerate}
	\item En Janvier 2020, on met de côté la somme de 10\,€ puis les mois suivants, on met de côté 2\,€ de plus que le mois précédent. On note $s_n$ la somme (en euros) mise de côté le $n$-ième mois. En particulier, on a $s_1=10$.
	\begin{enumerate}
		\item Déterminer les valeurs de $s_2$ et de $s_3$. Interpréter ces résultats dans le contexte de l'exercice.
		\item Déterminer, en justifiant, la nature de la suite $\suiten[s]$.
		\item En déduire une expression de $s_n$ en fonction de $n$.
		\item Déterminer la somme mise de côté en Décembre 2020, ainsi que la somme totale mise de côté en 2020.
	\end{enumerate}
	\item Une solution contient cinq bactéries à l'instant $t=0$. Après l'ajout d'un élément nutritif, le nombre de bactéries augmente de 25\,\% chaque seconde. On note $b_n$ le nombre de bactéries à l'instant $n$.
	\begin{enumerate}
		\item Préciser la valeur de $b_0$ puis calculer le nombre de bactéries au bout de 1 seconde puis de 2 secondes.
		\item Déterminer, en justifiant, la nature de la suite $\suiten[b]$.
		\item En déduire une expression de $b_n$ en fonction de $n$.
		\item Déterminer le nombre de bactérie au bout d'une minute.
		\item Déterminer, en détaillant, au bout de combien de temps le nombre de bactérie dépassera 500.
	\end{enumerate}
\end{enumerate}

\smallskip

\exonum{3}

\medskip

On considère la suite $\suiten$ définie par $u_0=65$ et, pour tout entier naturel $n$, $u_{n+1}=0,8u_n+18$.
\begin{enumerate}
	\item 
	\begin{enumerate}
		\item Calculer $u_1$ et $u_2$.
		\item Justifier que $\suiten$ n'est ni arithmétique, ni géométrique.
	\end{enumerate}
	\item Pour tout entier $n$, on pose $v_n=u_n-90$.
	\begin{enumerate}
		\item Calculer $v_0$, $v_1$ et $v_2$.
		\item Pour tout entier $n$, exprimer $v_{n+1}$ en fonction de $u_{n+1}$ puis en fonction de $u_n$.
		\item Justifier que, pour tout entier $n$, $v_{n+1}=0,8v_n$ puis en déduire la nature de la suite $\suiten[v]$.
		\item Exprimer $v_n$ en fonction de $n$.
	\end{enumerate}
	\item En déduire que $u_n=90-25 \times 0,8^n$ pour tout entier $n$.
	\item 
	\begin{enumerate}
		\item Déterminer la valeur de $u_{10}$.
		\item Déterminer, en justifiant, le sens de variation de la suite $\suiten$.
		\item Déterminer, en détaillant, le plus petit entier $n$ pour lequel $u_n \pg 85$.
	\end{enumerate}
\end{enumerate}

\smallskip

\exonum{5}

\medskip

On considère la suite $\suiten$ définie par $\begin{dcases} u_1 = \dfrac{1}{3} \\ u_{n+1} = \dfrac{n+1}{3n}u_n \text{ pour tout entier } n \pg 1 \end{dcases}$.
\begin{enumerate}
	\item Calculer $u_2$, $u_3$ et $u_4$.
	\item On pose, pour tout entier $n \pg 1$, $v_n = \dfrac{u_n}{n}$.
	\begin{enumerate}
		\item Montrer que, pour tout entier $n \pg 1$, $v_{n+1}=\dfrac{1}{3}v_n$.
		\item En déduire la nature de la suite $\suiten[v]$ puis en déduire une expression de $v_n$ en fonction de $n$.
	\end{enumerate}
	\item Montrer que $u_n= n \times \left( \dfrac13\right)^n$ pour tout entier $n \pg 1$.
	\item 
	\begin{enumerate}
		\item Montrer que $u_{n+1}-u_n = (1-2n) \left( \dfrac13\right)^{n+1}$ pour tout entier $n \pg 1$.
		\item En déduire le sens de variation de $\suiten$.
	\end{enumerate}
	\item Déterminer, en détaillant, le plus petit entier $n$ pour lequel $u_n \pp 10^{-6}$.
\end{enumerate}

\pagebreak

\exotitre{Compléments d'algorithmique}

\smallskip

\begin{pyconcode}
#terme
def terme_un(n):
	u = 500
	for i in range(1,n+1) :
		u = 0.95 * u + 20
	return u
	
	
def seuil_un(S) :
	n = 0
	u = 500
	while u >= S :
		u = 0.95 * u + 20
		n = n + 1
	return n,u
	
	
def somme_termes_un(n) :
	u = 500
	s = 500
	for i in range(2,n+1) :
		u = 0.95 * u + 20
		s = s + u
	return s
	
	
\end{pyconcode}

\begin{cexemple}
{\small On considère la suite $\suiten$ définie par $u_0=500$ et, pour tout entier $n$, $u_{n+1}=0,95u_n+20$. En \calgpython, on va : \vspace{-6pt}
\begin{center}
	\setlength\arrayrulewidth{1pt} \arrayrulecolor{black}
	\begin{tabularx}{\linewidth}{YYY}
		{\scriptsize\faPython} calculer un terme quelconque & {\scriptsize\faPython} déterminer un seuil & {\scriptsize\faPython} calculer une somme de termes \\
	\end{tabularx}
\end{center}}
\end{cexemple}

\begin{calgo}[ - Recherche d'un terme]
\vspace{-0.4cm}
\begin{envpython}[16cm]
def terme_un(n):
	u = 500                     # 1er terme de la suite
	for i in range(1,n+1) :     # i va de 1 à n
		u = 0.95 * u + 20       # formule de récurrence de (un)
	return u                    # on renvoie le terme d'indice n
\end{envpython}

\begin{envconsolepython}[16cm]
# calcul de u_7
terme_un(7)
\end{envconsolepython}
\end{calgo}

\begin{calgo}[ - Recherche d'un rang (algorithme de seuil)]
\vspace{-0.4cm}
\begin{envpython}[16cm]
def seuil_un(S) :
	n = 0                       # rang initial
	u = 500                     # 1er terme de la suite
	while u >= S :              # condition de sortie (ici u<S)
		u = 0.95 * u + 20       # le terme suivant via la formule de récurrence
		n = n + 1               # le rang suivant
	return n,u                  # on renvoie le rang (et le terme)
\end{envpython}

\begin{envconsolepython}[16cm]
# plus entier n tel que u_n < 450
seuil_un(450)
\end{envconsolepython}
\end{calgo}

\begin{calgo}[ - Recherche d'une somme]
\vspace{-0.4cm}
\begin{envpython}[16cm]
def somme_termes_un(n) :
	u = 500                     # le 1er terme vaut 500
	s = 500                     # la somme aussi
	for i in range(2,n+1) :     # i va de 2 à n
		u = 0.95 * u + 20       # formule de récurrence de (un)
		s = s + u               # on augmente la somme partielle de u
	return s                    # on affiche la somme complète
\end{envpython}

\begin{envconsolepython}[16cm]
# calcul de u_1 + u_2 + ... u_50
somme_termes_un(50)
\end{envconsolepython}
\end{calgo}

\begin{clog}
{\small Sur \cshell{\url{https://python.cpierquet.fr/?from=examples/2m2_algos_suites.py}} les scripts précédents peuvent être testés !}
\end{clog}

\end{document}