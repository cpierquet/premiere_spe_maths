% !TeX TXS-program:compile = txs:///arara
% arara: lualatex: {shell: no, synctex: yes, interaction: batchmode}
% arara: pythontex: {rerun: modified} if found('pytxcode', 'PYTHONTEX#py')
% arara: lualatex: {shell: no, synctex: yes, interaction: batchmode} if found('pytxcode', 'PYTHONTEX#py')
% arara: lualatex: {shell: no, synctex: yes, interaction: batchmode} if found('log', '(undefined references|Please rerun|Rerun to get)')

\documentclass[a4paper,11pt]{article}
\usepackage[revgoku]{cp-base}
\graphicspath{{./graphics/}}
%variables
\donnees[%
	classe=1\up*{ère} 2M2,
	matiere={[SPÉ.MATHS]},
	typedoc=TD,
	numdoc=08,
	mois=Juin,
	annee=2022
	]

%formatage
\author{Pierquet}
\title{\nomfichier}
\hypersetup{pdfauthor={Pierquet},pdftitle={\nomfichier},allbordercolors=white,pdfborder=0 0 0,pdfstartview=FitH}
%divers
\lhead{\entete{\matiere}}
\chead{\entete{\lycee}}
\rhead{\entete{\classe{} - \mois{} \annee}}
\lfoot{\pied{\matiere}}
\cfoot{\logolycee{}}
\rfoot{\pied{\numeropagetot}}

\begin{document}

\pagestyle{fancy}

\part{TD08 - Fonction exponentielle en situation}

Une start-up fabrique entre 100 et 2\,000 ordinateurs par jour. On admet que si la start-up fabrique $x$ \textbf{centaines d'ordinateurs}, le bénéfice en \textbf{centaines d'euros} est modélisé par : \[ f(x)=80x\,\e^{-0,2x} \text{, avec } x \in \intervFF{1}{20}.\]
%
On note $f'$ la dérivée de la fonction $f$, et on note $\mathscr{C}_f$ sa courbe représentative dans un repère orthogonal, donnée en ci-dessous.

\begin{center}
	\begin{tikzpicture}[x=0.6cm,y=0.06cm,xmin=0,xmax=21,xgrille=1,xgrilles=0.2,ymin=0,ymax=155,ygrille=10,ygrilles=2]
		\tgrilles \tgrillep \axestikz*
		\axextikz[size=\small]{0,1,...,20} \axeytikz[size=\small]{0,10,...,150}
		%\clip (\xmin,\ymin) rectangle (\xmax,\ymax) ;
		\draw[very thick,red,domain=1:20,samples=100] plot (\x,{80*\x*exp(-0.2*\x)}) ;
		\draw[red] (1.5,65) node {\large $\mathscr{C}_f$} ;
	\end{tikzpicture}
\end{center}

\textbf{\large Partie A -- Étude graphique}

\medskip

À l'aide du graphique et en laissant les traits de construction apparents :

\begin{enumerate}
	\item déterminer le maximum de la fonction $f$ sur l'intervalle $\intervFF{1}{20}$ ;
	\item résoudre l'équation $f(x)=100$, avec la précision permise par le graphique.
\end{enumerate}

\textbf{\large Partie B - Étude de la fonction $\grasmaths{f}$}

\begin{enumerate}
	\item 
	\begin{enumerate}
		\item Justifier que, pour tout $x$ appartenant $\intervFF{1}{20}$, on a $f'(x)=\e^{-0,2x}(80-16x)$.
		\item Étudier le signe de $f'(x)$ sur l'intervalle $\intervFF{1}{20}$.
		\item En déduire le tableau de variation de la fonction $f$ (les images seront, si besoin, arrondies au centième).
	\end{enumerate}
	\item Démontrer que l'équation $f(x) = 100$ admet une unique solution sur l'intervalle $\intervFF{1}{5}$ puis en déterminer, à l'aide de la calculatrice, une valeur approchée au centième.
	
	\smallskip
	
	On admet que sur l'intervalle $\intervFF{5}{20}$ l'équation $f(x)=100$ admet également une unique solution égale à environ $10,76$.
\end{enumerate}

\textbf{\large Partie C -- Interprétation}

\begin{enumerate}
	\item Déterminer le bénéfice maximal à l'euro près réalisé par la start-up et le nombre d'ordinateurs fabriqués pour le réaliser
	\item Entre quelles valeurs doit être compris le nombre d'ordinateurs fabriqués pour que la start-up réalise un bénéfice supérieur ou égal à 10\,000 euros ?
\end{enumerate}

\end{document}