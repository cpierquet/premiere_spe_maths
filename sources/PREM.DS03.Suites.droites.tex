% !TeX TXS-program:compile = txs:///pythonlualatex

\documentclass[a4paper,11pt]{article}
\usepackage[pythontex,sujet]{cp-base}
\graphicspath{{./graphics/}}
%variables
\donnees[
	classe={1\up{ère} 2M2},matiere={[SPÉ.MATHS]},mois={Mardi 23 Novembre},annee=2021,duree=1 heure,typedoc=DS,numdoc=3
]
%formatage
\author{Pierquet}
\title{\nomfichier}
\hypersetup{pdfauthor={Pierquet},pdftitle={\nomfichier},allbordercolors=white,pdfborder=0 0 0,pdfstartview=FitH}
%divers
\lhead{\entete{\matiere}}
\chead{\entete{\lycee}}
\rhead{\entete{\classe{} - \mois{} \annee}}
\lfoot{\pied{\matiere}}
\cfoot{\logolycee{}}
\rfoot{\pied{\numeropagetot}}
\fancypagestyle{enteteds}{\fancyhead[L]{\entete{Durée : \duree}}}

\begin{document}

\pagestyle{fancy}

\thispagestyle{enteteds}

\setcounter{numexos}{0}

\part{DS03 - Fonctions affines, suites numériques}

\smallskip

\nomprenomtcbox

\begin{marker}$\leftrightsquigarrow$ Le sujet est à rendre avec la copie. $\leftrightsquigarrow$\end{marker}

\exonum{7}%exo1

\begin{enumerate}
	\item On considère la droite $(d)$ d'équation $y=3x-12$.
	\begin{enumerate}
		\item Interpréter graphiquement les valeurs $3$ et $-12$ de l'équation $y=3x-12$.
		\item Vérifier que le point $A(5\,;\,3)$ appartient à la droite $(d)$.
		\item Déterminer, en justifiant, le sens de variation de la droite $(d)$.
		\item Déterminer le tableau de signes de l'expression $(3x-12)$.
	\end{enumerate}
	\item Dans le repère orthogonal suivant, on a tracé trois droites $(d_1)$, $(d_2)$ et $(d_3)$.
	\begin{center}
		\begin{tikzpicture}[x=1.2cm,y=1.2cm,xmin=-3,xmax=3,ymin=-3,ymax=3]
			\tgrilles[ultra thin,lightgray] \tgrillep[very thin,lightgray]
			\axestikz* \axextikz*{-3,-2,...,2} \axeytikz*{-3,-2,...,2}
			\clip (\xmin,\ymin) rectangle (\xmax,\ymax) ;
			\draw[red,line width=1.5pt,domain=\xmin:\xmax,samples=2] plot (\x,{2/3*\x-1}) ;
			\draw[blue,line width=1.5pt,domain=\xmin:\xmax,samples=2] plot (\x,{2}) ;
			\draw[ForestGreen,line width=1.5pt,domain=\xmin:\xmax,samples=2]  (-1.5,\ymin) -- (-1.5,\ymax) ;
			\draw (1,0) node[below=4pt] {$1$} (0,1) node[left=4pt] {$1$} (0,0) node[below left=4pt] {$0$} ;
			\draw[red] (2,0.75) node {\large $(d_1)$} ;
			\draw[blue] (2.5,2.25) node {\large $(d_2)$} ;
			\draw[ForestGreen] (-2,-1) node {\large $(d_3)$} ;
		\end{tikzpicture}
	\end{center}
	\begin{enumerate}
		\item Déterminer, en détaillant brièvement la démarche, une équation des droites $(d_1)$, $(d_2)$ et $(d_3)$.
		\item Tracer, dans le repère précédent, la droite $(d_4)$ d'équation $y=0,5x-2$.
	\end{enumerate}
	\item On considère les points K et L de coordonnées $K(10\,;\,8,95)$ et $L(-5,4\,;\,-2,6)$.
	\begin{enumerate}
		\item Déterminer, en détaillant la démarche, une équation de la droite $(KL)$. %y=0.75x+1.45
		\item Le point $P(15\,;\,12,5)$ appartient-il à la droite $(KL)$ ? Justifier la réponse.
	\end{enumerate}
\end{enumerate}

\medskip

\exonum{6}%exo2

\medskip

Déterminer le tableau de signes des expressions suivantes :
%
\begin{enumerate}
	\item $(3x-9)(10-2x)$ ;
	\item $\dfrac{1-x}{(2x+2)(4x-6)}$ ;
	\item $5+\dfrac{x-1}{x+1}$.
\end{enumerate}

\newpage

\exonum{4}%exo3

\medskip

On considère les suites $\suiten$ et $\suiten[v]$ définies par :
	
\begin{itemize}[leftmargin=4cm]
	\item $u_n = \sqrt{2n+5}$ pour tout entier naturel $n$ ;
	
	\medskip
	\item $\begin{dcases} v_1 = 1\,000 \\ v_{n+1} = 0,8v_n + 130 \text{ pour tout entier naturel } n \text{ non nul} \end{dcases}$.
\end{itemize}

\begin{enumerate}
	\item 
	\begin{enumerate}
		\item Calculer les valeurs de $u_0$, $u_1$, $u_2$ et $u_{10}$.
		\item Calculer, en détaillant un minimum, les valeurs de $v_2$, $v_3$ et $v_{5}$.
	\end{enumerate}
	\item Répondre aux questions suivantes, sans justifier, et en utilisant la calculatrice :
	\begin{enumerate}
		\item déterminer une valeur approchée de $v_{20}$ au centième ;
		\item conjecturer le sens de variations de la suite $\suiten[v]$ ;
		\item conjecturer la limite éventuelle de la suite $\suiten[v]$.
	\end{enumerate}
\end{enumerate}

\medskip

\exonum{3}%exo4

\medskip

On considère une ville pour laquelle la population augmente de 3\,\% par an.

\smallskip

En 2018, la population de la ville était de 10\,000 habitants.

\smallskip

On note $\suiten[P]$ la population de cette ville l'année $(2018+n)$. On a donc $P_0=10\,000$.

\begin{enumerate}
	\item Calculer $P_1$ et $P_2$. Interpréter ces résultats dans le contexte de l'exercice.
	\item On considère la feuille de tableur suivante :
	
	\begin{center}
		\begin{tikzpicture}
			\tableur*[4]{A/3cm,B/3cm}
			%L1
			\celtxt[c]{A}{1}{n}
			\celtxt[c]{B}{1}{P\_n}
			%L2
			\celtxt[c]{A}{2}{0}
			\celtxt[c]{B}{2}{10\,000}
			%L3
			\celtxt[c]{A}{3}{1}
			%L4
			\celtxt[c]{A}{4}{2}
		\end{tikzpicture}
	\end{center}
	Quelle formule est à rentrer dans la cellule \csheet{B3} afin de calculer, par recopie, les termes de $\suiten[P]$ ?
	\item En utilisant la calculatrice, et en détaillant un minimum :
	\begin{enumerate}
		\item déterminer la population (estimée) de la ville en 2032 ;
		\item déterminer en quelle année la population aura doublé.
	\end{enumerate}
\end{enumerate}

\medskip

\exonum[ Bonus]{1}

\medskip

On considère l'algorithme suivant en \calgpython{} :

\begin{tcpythoncode}[15cm]
	\begin{pyverbatim}[][fontsize=\footnotesize,numbers=left,numbersep=10pt]
		def signe(m,p) :
			if m > 0 :
				racine , signe = -p/m , '-0+'
				print(f"mx+p s'annule en {racine} et son signe est {signe}")
			if m < 0 :
				racine , signe = -p/m , '+0-'
				print(f"mx+p s'annule en {racine} et son signe est {signe}")
	\end{pyverbatim}
\end{tcpythoncode}

Que renverra l'appel \cpy{signe(-5,6)} ?


\end{document}