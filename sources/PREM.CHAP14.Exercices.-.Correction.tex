% !TeX TXS-program:compile = txs:///arara
% arara: lualatex: {shell: no, synctex: yes, interaction: batchmode}
% arara: pythontex: {rerun: modified} if found('pytxcode', 'PYTHONTEX#py')
% arara: lualatex: {shell: no, synctex: yes, interaction: batchmode} if found('pytxcode', 'PYTHONTEX#py')
% arara: lualatex: {shell: no, synctex: yes, interaction: batchmode} if found('log', '(undefined references|Please rerun|Rerun to get)')

\documentclass[a4paper,11pt]{article}
\usepackage[revgoku]{cp-base}
\graphicspath{{./graphics/}}
%variables
\donnees[classe={1\up*{ère} 2M2},matiere={[SPÉ.MATHS]},mois=Juin,annee=2022,typedoc=CHAP,numdoc=14]

%formatage
\author{Pierquet}
\title{\nomfichier}
\hypersetup{pdfauthor={Pierquet},pdftitle={\nomfichier},allbordercolors=white,pdfborder=0 0 0,pdfstartview=FitH}
%fancy
\lhead{\entete{\matiere}}
\chead{\entete{\lycee}}
\rhead{\entete{\classe{} - \mois{} \annee}}
\lfoot{\pied{\matiere}}
\cfoot{\logolycee{}}
\rfoot{\pied{\numeropagetot}}

\begin{document}

\pagestyle{fancy}

\part{CH14 - Compléments sur le produit scalaire - Exercices (Correction)}

\exonum{1}

\begin{enumerate}
	\item 
	\begin{enumerate}
		\item Pour déterminer un point de cette droite, on peut fixer une valeur de $x$ et déterminer la valeur de $y$.
		
		De plus, le vecteur $\vect{u} \begin{pmatrix} -b\\a \end{pmatrix}$ est un vecteur directeur de la droite d'équation $ax+by+c=0$.
		
		\begin{itemize}[leftmargin=*]
			\item pour $(d_1)$, on prend $x=0$ et on trouve $-3y+3=0 \ssi y=1$ et on obtient directement $\vect{u} \begin{pmatrix} 3\\2\end{pmatrix}$ ;
			\item pour $(d_2)$, on prend $x=0$ et on trouve $-y+1=0 \ssi y=1$ et on obtient directement $\vect{u} \begin{pmatrix} 1\\-2\end{pmatrix}$ ;
			\item pour $(d_3)$, on prend $x=0$ et on trouve $8y-10=0 \ssi y=\tfrac{10}{8}=\tfrac{5}{4}$ et on obtient directement $\vect{u} \begin{pmatrix} -8\\4\end{pmatrix}$ ;
			\item pour $(d_4)$, on prend $x=0$ et on trouve $y+4=0 \ssi y=-4$ et on obtient directement $\vect{u} \begin{pmatrix} -1\\-3\end{pmatrix}$.
		\end{itemize}
		\item On peut \og repasser \fg{} par l'équation réduite pour tracer les droites :
		
		\begin{itemize}
			\item $\mathcolor{red}{(d_1)}$ : $2x − 3y + 3 = 0 \ssi y=\tfrac{2}{3}x+1$ ;
			\item $\mathcolor{blue}{(d_2)}$ : $-2x - y + 1 = 0 \ssi y=2x+1$ ;
			\item $\mathcolor{purple}{(d_3)}$ : $4x + 8y - 10 = 0  \ssi y=-\tfrac{4}{8}x+\tfrac{10}{8}=-\tfrac12 x + \tfrac{5}{4}$ ;
			\item $\mathcolor{ForestGreen}{(d_4)}$ : $-3x + y + 4 = 0 \ssi y = 3x-4$.
		\end{itemize}
		
		\begin{center}
			\begin{tikzpicture}[x=0.8cm,y=0.8cm,xmin=-4,xmax=4,xgrille=1,xgrilles=0.25,ymin=-3,ymax=3,ygrille=1,ygrilles=0.25]
				\tgrilles \tgrillep \axestikz*
				\axextikz[size=\small]{-4,-3,-2,-1,1,2,3}
				\axeytikz[size=\small]{-3,-2,-1,1,2}
				\draw (0,0) node[below left=2pt] {$0$} ;
				\clip (\xmin,\ymin) rectangle (\xmax,\ymax) ;
				\draw[red,very thick,domain=\xmin:\xmax,samples=2] plot (\x,{2/3*\x+1}) ;
				\draw[blue,very thick,domain=\xmin:\xmax,samples=2] plot (\x,{2*\x+1}) ;
				\draw[purple,very thick,domain=\xmin:\xmax,samples=2] plot (\x,{-0.5*\x+1.25}) ;
				\draw[ForestGreen,very thick,domain=\xmin:\xmax,samples=2] plot (\x,{3*\x-4}) ;
			\end{tikzpicture}
		\end{center}
	\end{enumerate}
	\item 
	\begin{enumerate}
		\item $A(2\,;\,1)$ et $\vect{u} \begin{pmatrix}2\\3\end{pmatrix}$ :
		\begin{itemize}
			\item une équation cartésienne est de la forme $3x-2y+c=0$ ;
			\item avec le point $A$, on trouve $3\times2-2\times1+c=0 \ssi c=-4$.
		\end{itemize}
		On obtient $3x+2y-4=0$.
		\item Pour $A(3\,;\,-2)$ et $\vect{u} \begin{pmatrix}\tfrac12\\-1\end{pmatrix}$.
		
		On a $-x-\tfrac12 y+c=0$ et $-3 -\tfrac12 \times (-2) + c = 0 \ssi c=2$.
		
		On obtient $-x-\tfrac12 y + 2=0 \ssi -2x-y+4=0 \ssi 2x+y-4=0$.
		\item Pour $A(0\,;\,3)$ et $\vect{u} \begin{pmatrix}-2\\1\end{pmatrix}$.
		
		On a $x+2y+c=0$ et $0+2\times3+c=0 \ssi c=-6$.
		
		On obtient $x+2y-6=0$.
		\item Pour $A\left(-2\,;\,\tfrac12\right)$ et $\vect{u} \begin{pmatrix}3\\-\tfrac53\end{pmatrix}$.
		
		On a $-\tfrac53 x-3y+c=0$ et $-\tfrac53 \times (-2)-3\times\tfrac12+c=0 \ssi c=-\tfrac{11}{6}$.
		
		On obtient $-\tfrac53 x-3y-\tfrac{11}{6}=0 \ssi -10x-18y-11=0 \ssi 10x+18y-11=0$.
	\end{enumerate}
	\item Le vecteur $\vect{n} \begin{pmatrix} a\\b \end{pmatrix}$ est un vecteur normal de la droite d'équation $ax+by+c=0$. Donc 
	\begin{enumerate}
		\item $\vect{n} \begin{pmatrix}2\\3\end{pmatrix}$ et $A(1\,;\,0)$ donnent $2x+3y+c=0$ puis $2\times1+3\times0+c=0 \ssi c=-2$.
		
		On obtient l'équation cartésienne $2x+3y-2=0$.
		\item $\vect{n} \begin{pmatrix}-1\\1\end{pmatrix}$ et $A(-2\,;\,1)$ donnent $-x+y+c=0$ puis $-(-2)+1+c=0 \ssi c=-3$.
		
		On obtient l'équation cartésienne $-x+y-3=0$.
	\end{enumerate}
	\item 
	\begin{enumerate}
		\item On considère la droite $(d)$ admettant le vecteur $\vect{n} \begin{pmatrix}-2\\1\end{pmatrix}$ pour vecteur normal et passant par  $A(4\,;\,1)$.
		
		Une équation cartésienne de la droite $(d)$ est $-2x+y+c=0$ avec $-2\times4+1+c=0 \ssi c=7$.
		
		Ainsi on obtient $(d)$ : $-2x+y+7=0$.
		\item Pour la droite $(D)$ : $x - 4y + 3 = 0$ :
		
		\begin{itemize}
			\item $\vect{v'} \begin{pmatrix} 1\\-4 \end{pmatrix}$ est un vecteur normal de $(D)$ ;
			\item $\vect{u} \begin{pmatrix} 4\\1 \end{pmatrix}$ est un vecteur directeur de $(D)$ ;
			\item avec $x=1$, on trouve $y=1$ ce qui donne le point $B(1\,;\,1)$.
		\end{itemize}
	\end{enumerate}
\end{enumerate}

\medskip

\exonum{1}

\begin{enumerate}
	\item On obtient directement $K(1\,;\,-1)$.
	\item La médiatrice du segment $[AB]$ étant une droite perpendiculaire à la droite $(AB)$, on en déduit que le vecteur $\vect{AB} \begin{pmatrix} 4-(-2) \\ 1-(-3) \end{pmatrix} = \begin{pmatrix} 6 \\ 4 \end{pmatrix}$ est un vecteur normal à la droite $(d)$.
	\item On en déduit que la droite $(d)$ admet pour équation cartésienne : $6x + 4y + c = 0$.
	
	Le point $K$ appartenant à la droite $(d)$ : $6\times1 + 4\times(-1) + c = 0 \ssi c = -2$.
	
	Ainsi la droite $(d)$ admet pour équation cartésienne : $6x + 4y - 2 = 0$.
\end{enumerate}

\begin{center}
	\begin{tikzpicture}[x=1cm,y=1cm,xmin=-3,xmax=5,xgrille=1,xgrilles=0.5,ymin=-4,ymax=2,ygrille=1,ygrilles=0.5]
		\tgrilles \tgrillep \axestikz* \axextikz*{-3,-2,...,4} \axeytikz*{-4,-3,...,1}
		\clip (\xmin,\ymin) rectangle (\xmax,\ymax) ;
		\filldraw (-2,-3) circle[radius=2pt] node[below] {$A$} ;
		\filldraw (4,1) circle[radius=2pt] node[above] {$B$} ;
		\draw[very thick] (-2,-3)--(4,1) ;
		\filldraw[blue] (1,-1) circle[radius=2pt] node[above] {$K$} ;
		\draw[red,very thick,densely dashed,domain=\xmin:\xmax,samples=2] plot (\x,{(-6*\x+2)/4}) ;
		\draw[red] (3,-4) node[above right] {$(d)$} ;
	\end{tikzpicture}
\end{center}

\medskip

\exonum{1}

\begin{enumerate}
	\item Un vecteur $\vect{u} \begin{pmatrix} -1 \\ -3 \end{pmatrix}$ est directeur de la droite $(d)$.
	\item La droite $(\Delta)$ est parallèle à $(d)$, donc a une équation de la forme $-3x+y+c=0$ (même vecteur directeur).
	
	Avec le point $A$, on a $-3\times(-2) + 2 + c = 0 \ssi c = -8$. Ainsi $(\Delta)$ : $-3x + y - 8 = 0$.
	\item 
	\begin{enumerate}
		\item Le vecteur $\vect{v}$ étant normal à la droite $(D)$ et la droite $(D)$ étant perpendiculaire à la droite $(d)$, on en déduit que $\vect{v}$ est un vecteur directeur de la $(d)$.
		
		De ce fait, on peut prendre $\begin{pmatrix} -1 \\ -3 \end{pmatrix}$ ou $\begin{pmatrix} 1 \\ 3 \end{pmatrix}$ pour vecteur normal à $(D)$.
		\item La droite $(D)$ admet une équation cartésienne de la forme : $x + 3y + c = 0$.
		
		Avec le point $B$, on obtient $3 + 3\times(-1) + c = 0 \ssi c = 0$.
		
		Ainsi la droite $(D)$ admet pour équation cartésienne : $x + 3y = 0$.
	\end{enumerate}
\end{enumerate}

\medskip

\exonum{2}

%\medskip
%
%Dans le plan, on considère les deux carrés $ABCD$ et $BEFG$ représentés ci-dessous :
%
%\begin{center}
%	\begin{tikzpicture}[]
%		\tkzDefPoints{0/0/A,6/0/B,6/6/C,0/6/D,9/0/E,9/-3/F,6/-3/G}
%		\draw[thick] (A) rectangle (C) ;
%		\draw[thick] (B) rectangle (F) ;
%		\draw[thick,->] (A)--++(1,0) node[midway,below] {$\vect{\imath}$} ;
%		\draw[thick,->] (A)--++(0,1) node[midway,left] {$\vect{\jmath}$} ;
%		\foreach \Point/\Pos in {A/below left,B/below left,C/above right,D/above,E/above right,F/below right,G/below left}
%			{\tkzLabelPoint[\Pos](\Point){\Point}}
%		\tkzDrawLines[thick,dashed,add=.1 and .1](D,E)
%		\tkzDrawLines[thick,dashed,add=.1 and .1](C,F)
%		\tkzDrawLines[thick,dashed,add=.1 and .1](C,E)
%		\tkzInterLL(C,F)(D,E)\tkzGetPoint{I}
%		\tkzLabelPoint[below](I){I}
%		\tkzDrawLines[thick,dashed,add=.1 and .5](B,I)
%	\end{tikzpicture}
%\end{center}
%
%On munit le plan du repère orthonormé $\left(A\,;\,\vect{i},\vect{j}\right)$ où $AB=6$ et $BE=3$.

\begin{enumerate}
	\item 
	\begin{enumerate}
		\item On lit directement $\vect{DE} \begin{pmatrix} 9\\-6 \end{pmatrix}$ et $\vect{CF} \begin{pmatrix} 3\\-9 \end{pmatrix}$.
		\item On détermine les équations cartésiennes des droites :
		
		\begin{itemize}[leftmargin=*]
			\item $\vect{DE}$ est un vecteur directeur de la droite $(DE)$, son équation cartésienne est de la forme : $6x + 9y + c = 0$ ;
			
			avec le point $D$, on a $6\times0 + 9\times6 + c = 0 \ssi c = -54$ ;
			
			une équation cartésienne de la droite $(DE)$ est : $6x + 9y - 54 = 0$.
			\item $\vect{FC}$ est un vecteur directeur de la droite $(FC)$, son équation cartésienne est de la forme : $9x + 3y + c = 0$ ;
			
			avec le point $C$, on a $9\times6 + 3\times6 + c = 0 \ssi c = -72$ ;
			
			une équation cartésienne de la droite $(FC)$ est : $9x + 3y - 72 = 0$.
		\end{itemize}
	\end{enumerate}
	\item Le point I est l’intersection de $(DE)$ et $(CF)$, ses coordonnées doivent vérifier le système $\begin{dcases} 6x + 9y - 54 = 0 \\ 9x + 3y - 72 = 0 \end{dcases}$.
	
	Par substitution, ou par calculatrice, on obtient $I \left( \dfrac{54}{7}\,;\,\dfrac{6}{7} \right)$.
	\item On travaille sur les vecteurs $\vect{BI} \begin{pmatrix} \nicefrac{12}{7} \\ \tfrac{6}{7} \end{pmatrix}$ et $\vect{CE} \begin{pmatrix} 3\\-6 \end{pmatrix}$. On a donc $\vect{BI}\cdot\vect{CE} = \frac{12}{7}\times3+\frac{6}{7}\times(-6)=\frac{36}{7}-\frac{36}{7}=0$.
	
	On en déduit que les vecteurs $\vect{BI}$ et $\vect{CE}$ sont orthogonaux, et donc que $(BI)$ et $(CE)$ sont perpendiculaires.
\end{enumerate}

\medskip

\exonum{2}

\begin{enumerate}
	\item 
	\begin{enumerate}
		\item Avec $I(1\,;\,2)$ et $r=3$~cm, on obtient $(x-1)^2+(y-2)^2=3^2$.
		
		Soit encore $x^2-2x+1+y^2-4y+4=9 \ssi x^2 + y^2 - 2x - 4y + 4 = 0$.
		\item Avec $I(-3\,;\,1)$ et $r=5$~cm, on obtient $(x-(-3))^2+(y-1)^2=5^2$.
		
		Soit encore $x^2+6x+9+y^2-2y+1=25 \ssi x^2+y^2+6x-2y-15=0$.
	\end{enumerate}
	\item Une équation cartésienne de $\mathscr{C}$ est $(x-2)^2+(y-1)^2=4^2$.
	
	Soit encore $x^2-4x+4+y^2-2y+1=16 \ssi x^2+y^2-4x-2y-11=0$.
	\item 
	\begin{enumerate}
		\item Une équation cartésienne du cercle $\mathscr{C}$ est $(x-3)^2+(y-(-1))^2=5^2$.
		
		Soit encore $x^2-6x+9+y^2+2y+1=25 \ssi x^2+y^2-6x+2y-15=0$.
		\item On teste l'équation de $\mathscr{C}$ avec les coordonnées des points proposés :
		
		\begin{itemize}
			\item $x_M^2+y_M^2-6x_M+2y_M-15=(-1)^2 + 2^2 − 6\times(-1) + 2\times2 - 15 = 1 + 4 + 6 + 4 - 15 = 0$, donc $M \in \mathscr{C}$.
		
			\item $x_N^2 + y_N^2 - 6x_N + 2y_N - 15 = \ldots = 0$, donc $N  \in \mathscr{C}$.
			\item $x_P^2 + y_P^2 - 6x_P + 2y_P - 15 = \ldots = -21,6 \neq 0$, donc $P \not\in \mathscr{C}$.
		\end{itemize}
	\end{enumerate}
\end{enumerate}

\pagebreak

\exonum{3}

\begin{center}
	\begin{tikzpicture}[x=0.8cm,y=0.8cm,xmin=-4,xmax=4,xgrille=1,xgrilles=0.25,ymin=-2,ymax=3,ygrille=1,ygrilles=0.25]
		\tgrilles \tgrillep \axestikz*
		\axextikz[size=\small]{-4,-3,-2,-1,1,2,3}
		\axeytikz[size=\small]{-1,1,2}
		\draw (0,0) node[below left=2pt] {$0$} ;
		\draw[thick,->] (0,0)--++(1,0) node[midway,below,font=\small] {$\vect{\imath}$} ;
		\draw[thick,->] (0,0)--++(0,1) node[midway,left,font=\small] {$\vect{\jmath}$} ;
		\clip (\xmin,\ymin) rectangle (\xmax,\ymax) ;
		\draw[very thick,red,domain=\xmin:\xmax,samples=2] plot (\x,{-2/5*\x+2/5}) ;
		\tkzDefPoints{1/0/R,2.3793/-0.551/S,3/1/T}
		\tkzMarkRightAngle[size=0.25,thick](T,S,R)
		\draw[red] (-3,2) node {$(d)$} ;
		\draw[blue] (3,2) node {$(\Delta)$} ;
		\filldraw (3,1) circle[radius=2pt] node[right] {$A$} ;
		\draw[very thick,blue,densely dashed,domain=\xmin:\xmax,samples=2] plot (\x,{(5*\x-13)/2}) ;
		\filldraw[ForestGreen] ({69/29},{-16/29}) circle[radius=2pt] node[below] {$H$} ;
	\end{tikzpicture}
\end{center}

\begin{enumerate}
	\item Pour construire $H$, on trace la perpendiculaire à $(d)$ passant par $A$.
	\item La droite $(d)$ a pour vecteur directeur $\vect{u} \begin{pmatrix} -5\\2 \end{pmatrix}$, qui est donc un vecteur normal à $(\Delta)$.
	
	Ainsi $(\Delta)$ a pour équation cartésienne : $-5x+2y+c=0$ et avec $A$, on a $-5\times3+2\times1+c=0 \ssi c=13$.
	
	On (re)trouve bien comme équation cartésienne de $(\Delta)$ : $-5x + 2y + 13 = 0$.
	\item Le point $H$ étant l'intersection de $(d)$ et de $(\Delta)$, donc on s'intéresse au système $\begin{dcases} 2x+5y-2=0 \\ -5x + 2y + 13 = 0 \end{dcases}$.
	
	Par substitution, ou par calculatrice, on obtient $H \left( \dfrac{69}{29}\,;\,-\dfrac{16}{29} \right)$.
\end{enumerate}

\medskip

\exonum{3}

%\medskip
%
%Dans le plan muni d’un repère $\Rij$, on considère les trois points : $A(-3\,;\,2)$ ; $B(3\,;\,5)$ ; $C(2\,;\,2)$.

\begin{enumerate}
	\item Le point $H$ est l'intersection de la droite $(AB)$ et de la perpendiculaire, notée $(d)$, à $(AB)$ passant par $C$.
	
	Or $\vect{AB} \begin{pmatrix} 6\\3 \end{pmatrix}$ est directeur de $(AB)$ et normal à $(d)$.
	
	\begin{itemize}[leftmargin=*]
		\item $(AB)$ : $3x-6y+c=0$ et avec $A$ on trouve $3\times(-3)-6\times2+c=0 \ssi c=21$.
		
		Ainsi $(AB)$ : $3x-6y+21=0 \ssi x-2y+7=0$.
		\item $(d)$ : $6x+3y+c=0$ et avec $C$ on trouve $6\times2+3\times2+c=0 \ssi c=-18$
		
		Ainsi $(d)$ : $6x+3y-18=0 \ssi 2x+y-6=0$.
	\end{itemize}
	
	Les coordonnées de $H$ vérifient le système $\begin{dcases} x-2y+7=0 \\  2x+y-6=0 \end{dcases}$, on obtient $H(1\,;\,4)$.
	\item L'aire du triangle $ABC$. peut s'obtenir de ce fait par $\mathscr{A}=\dfrac{AB\times CH}{2}$ :
	
	$\left. \begin{dcases} AB=\sqrt{6^2+3^2}=\sqrt{45} \\ CH=\sqrt{(1-2)^2+(4-2)^2}=\sqrt{(-1)^2+2^2}=\sqrt{5} \end{dcases} \right| \Rightarrow \mathscr{A}=\dfrac{AB\times CH}{2}=\dfrac{\sqrt{45}\times\sqrt{5}}{2}=\dfrac{15}{2}=7,5$ unités d'aires.
\end{enumerate}

\begin{center}
	\begin{tikzpicture}[x=0.8cm,y=0.8cm,xmin=-4,xmax=4,xgrille=1,xgrilles=0.25,ymin=0,ymax=6,ygrille=1,ygrilles=0.25,line join=bevel]
		\tgrilles \tgrillep \axestikz*
		\axextikz[size=\small]{-4,-3,-2,-1,0,1,2,3}
		\axeytikz[size=\small]{0,1,2,3,4,5}
		\tkzDefPoints{-3/2/A,3/5/B,2/2/C,1/4/H}
		\tkzMarkRightAngle[size=0.25,thick](A,H,C)
		\draw[very thick,blue] (A)--(B)--(C)--cycle ;
		\draw[very thick,red,densely dashed] (C)--(H) ;
		\draw[blue] (A) node[left] {$A$} (B) node[above right] {$B$} (C) node[below] {$C$} ;
		\draw[red] (H) node[above] {$H$} ;
%		\draw[thick,->] (0,0)--++(1,0) node[midway,below,font=\small] {$\vect{\imath}$} ;
%		\draw[thick,->] (0,0)--++(0,1) node[midway,left,font=\small] {$\vect{\jmath}$} ;
%		\clip (\xmin,\ymin) rectangle (\xmax,\ymax) ;
%		\draw[very thick,red,domain=\xmin:\xmax,samples=2] plot (\x,{-2/5*\x+2/5}) ;
%		\tkzDefPoints{1/0/R,2.3793/-0.551/S,3/1/T}
%		\tkzMarkRightAngle[size=0.25,thick](T,S,R)
%		\draw[red] (-3,2) node {$(d)$} ;
%		\draw[blue] (3,2) node {$(\Delta)$} ;
%		\filldraw (3,1) circle[radius=2pt] node[right] {$A$} ;
%		\draw[very thick,blue,densely dashed,domain=\xmin:\xmax,samples=2] plot (\x,{(5*\x-13)/2}) ;
%		\filldraw[ForestGreen] ({69/29},{-16/29}) circle[radius=2pt] node[below] {$H$} ;
	\end{tikzpicture}
\end{center}

\end{document}