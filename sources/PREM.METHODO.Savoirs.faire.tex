% !TeX TXS-program:compile = txs:///lualatex

\documentclass[a4paper,11pt]{article}
\usepackage[]{cp-base}
\graphicspath{{./graphics/}}
%variables
\donnees[classe=1\up*{ère} 2M2,matiere={[SPÉ.MATHS]},titre=Méthodes et techniques à connaître,annee=2021/2022,typedoc=CHAPITRE~,numdoc=15]

%formatage
\author{Pierquet}
\title{\nomfichier}
\hypersetup{pdfauthor={Pierquet},pdftitle={\nomfichier},allbordercolors=white,pdfborder=0 0 0,pdfstartview=FitH}
%divers
\lhead{\entete{\matiere}}
\chead{\entete{\lycee}}
\rhead{\entete{\classe{} - \annee}}
\lfoot{\pied{\matiere}}
\cfoot{\logolycee{}}
\rfoot{\pied{\numeropagetot}}

\begin{document}

%\renewtcolorbox{calgo}[1][]{%
%	cdefaut,%
%	colframe=OrangeRed,coltitle=black,colbacktitle=OrangeRed!35,colback=White,
%	title=\textcolor{OrangeRed!50!Black}{\faFileCode}\:\:\textbf{Algorithme#1}
%}
\newcommand\titrexo[1]{\textbf{\textcolor{titrebleu}{\Large\sffamily #1}}}

\pagestyle{fancy}

\part{CH15 - Méthodes et techniques à connaître}

\medskip

\makeatletter
	\def\hrulefill{\leavevmode\leaders\hrule height 1.1pt\hfill\kern\z@}%épaisseur
\makeatother

\titrexo{Chapitre 01 : Polynômes du second degré}\color{titrebleu}\,\hrulefill\,\titrexo{($\rightsquigarrow$ MC)}

\begin{casavoir}
\fontfamily{lmss}\selectfont\small
\textbf{Je sais :}

\tabula{}$\bullet~~$reconnaître les différentes formes (développée, canonique, factorisée)\dotfill{}$\Large\Box$

\tabula{}$\bullet~~$passer d'une forme à une autre\dotfill{}$\Large\Box$

\tabula{}$\bullet~~$déterminer le tableau de variations d'un trinôme\dotfill{}$\Large\Box$

\tabula{}$\bullet~~$calculer le discriminant et les éventuelles racines d'un trinôme\dotfill{}$\Large\Box$

\tabula{}$\bullet~~$résoudre une équation du 2\up{d} degré\dotfill{}$\Large\Box$

\tabula{}$\bullet~~$déterminer l'équation d'une parabole connaissant certaines de ses caractéristiques\dotfill{}$\Large\Box$
\end{casavoir}

\begin{ccalco}
\fontfamily{lmss}\selectfont\small
Je peux, avec ma calculatrice :

\tabula{}$\bullet~~$déterminer les solutions d'une équation du 2\up{d} degré avec le menu équation\dotfill{}$\Large\Box$
\end{ccalco}

\titrexo{Chapitre 02 : Généralités sur les suites numériques}\color{titrebleu}\,\hrulefill\,\titrexo{($\rightsquigarrow$ MC)}

\begin{casavoir}
\fontfamily{lmss}\selectfont\small
\textbf{Je sais :}

\tabula{}$\bullet~~$travailler avec une formule explicite, une formule de récurrence\dotfill{}$\Large\Box$

\tabula{}$\bullet~~$représenter graphiquement les termes d'une suite (nuage, toile)\dotfill{}$\Large\Box$

\tabula{}$\bullet~~$étudier le sens de variation (ou la monotonie) d'une suite\dotfill{}$\Large\Box$

\tabula{}$\bullet~~$conjecturer la limite éventuelle d'une suite\dotfill{}$\Large\Box$
\end{casavoir}

\begin{ccalco}
\fontfamily{lmss}\selectfont\small
Je peux, avec ma calculatrice :

\tabula{}$\bullet~~$afficher les premiers termes d'une suite, grâce au menu suite\dotfill{}$\Large\Box$

\tabula{}$\bullet~~$conjecturer variations, limite\dotfill{}$\Large\Box$

\tabula{}$\bullet~~$tabuler pour déterminer un seuil\dotfill{}$\Large\Box$
\end{ccalco}

\begin{coutil}
\fontfamily{lmss}\selectfont\small
Je peux, avec un tableur :

\tabula{}$\bullet~~$déterminer les formules nécessaires à l'étude de suites\dotfill{}$\Large\Box$

\tabula{}$\bullet~~$conjecturer variations, limite\dotfill{}$\Large\Box$

\tabula{}$\bullet~~$tabuler pour déterminer un seuil\dotfill{}$\Large\Box$
\end{coutil}

\begin{calgo}
\fontfamily{lmss}\selectfont\small
Je peux, avec un \calg{algorithme} :

\tabula{}$\bullet~~$déterminer le terme d'une suite récurrente\dotfill{}$\Large\Box$

\tabula{}$\bullet~~$déterminer un seuil\dotfill{}$\Large\Box$

\tabula{}$\bullet~~$calculer une somme de termes\dotfill{}$\Large\Box$
\end{calgo}

\pagebreak

\titrexo{Chapitre 03 : Droites et fonctions affines}\color{titrebleu}\,\hrulefill\,\titrexo{($\rightsquigarrow$ MC)}

\begin{casavoir}
\fontfamily{lmss}\selectfont\small
\textbf{Je sais :}

\tabula{}$\bullet~~$tracer une droite connaissant son équation réduite\dotfill{}$\Large\Box$

\tabula{}$\bullet~~$travailler graphiquement sur la pente et l'ordonnée à l'origine\dotfill{}$\Large\Box$

\tabula{}$\bullet~~$déterminer, algébriquement, l'équation d'une droite passant par deux points\dotfill{}$\Large\Box$

\tabula{}$\bullet~~$déterminer le (tableau de) signe et la variation d'une fonction affine\dotfill{}$\Large\Box$
\end{casavoir}

\titrexo{Chapitre 04 : Second degré, études de signes}\color{titrebleu}\,\hrulefill\,\titrexo{($\rightsquigarrow$ MC)}

\begin{casavoir}
\fontfamily{lmss}\selectfont\small
\textbf{Je sais :}

\tabula{}$\bullet~~$déterminer le (tableau de) signe d'un trinôme\dotfill{}$\Large\Box$

\tabula{}$\bullet~~$étudier un signe ou résoudre une inéquation avec un tableau de signes (ZPQ)\dotfill{}$\Large\Box$

\tabula{}$\bullet~~$étudier la position relative de deux courbes\dotfill{}$\Large\Box$
\end{casavoir}

\begin{ccalco}
\fontfamily{lmss}\selectfont\small
Je peux, avec ma calculatrice :

\tabula{}$\bullet~~$vérifier graphiquement les solutions d'une inéquation\dotfill{}$\Large\Box$
\end{ccalco}

\titrexo{Chapitre 05 : Probabilités conditionnelles, indépendance}\color{titrebleu}\,\hrulefill\,\titrexo{($\rightsquigarrow$ MC)}

\begin{casavoir}
\fontfamily{lmss}\selectfont\small
\textbf{Je sais :}

\tabula{}$\bullet~~$utiliser la formule de la réunion\dotfill{}$\Large\Box$

\tabula{}$\bullet~~$travailler sur l'indépendance de deux évènements\dotfill{}$\Large\Box$

\tabula{}$\bullet~~$modéliser une situation à l'aide d'un tableau, d'un arbre, d'un diagramme, etc\dotfill{}$\Large\Box$

\tabula{}$\bullet~~$utiliser un tableau pour calculer des probas (intersection, réunion, conditionnelles)\dotfill{}$\Large\Box$

\tabula{}$\bullet~~$utiliser un arbre pour calculer des probas (composées, totales, conditionnelles)\dotfill{}$\Large\Box$
\end{casavoir}

\titrexo{Chapitre 06 : Trigonométrie}\color{titrebleu}\,\hrulefill

\begin{casavoir}
\fontfamily{lmss}\selectfont\small
\textbf{Je sais :}

\tabula{}$\bullet~~$passer des radians aux degrés\dotfill{}$\Large\Box$

\tabula{}$\bullet~~$manipuler le cercle trigonométrique, y placer un angle en radian\dotfill{}$\Large\Box$

\tabula{}$\bullet~~$lire un angle associé à un point du cercle trigonométrique\dotfill{}$\Large\Box$

\tabula{}$\bullet~~$manipuler les cosinus et sinus des angles particuliers\dotfill{}$\Large\Box$

\tabula{}$\bullet~~$résoudre des équations trigonométriques simples\dotfill{}$\Large\Box$
\end{casavoir}

\begin{ccalco}
\fontfamily{lmss}\selectfont\small
Je peux, avec ma calculatrice :

\tabula{}$\bullet~~$convertir des degrés aux radians, et inversement\dotfill{}$\Large\Box$

\tabula{}$\bullet~~$déterminer le cosinus, le sinus et la tangente d'un réel\dotfill{}$\Large\Box$
\end{ccalco}

\begin{coutil}
\fontfamily{lmss}\selectfont\small
Je peux, avec mon cercle trigonométrique :

\tabula{}$\bullet~~$lire les lignes trigonométriques d'un angle\dotfill{}$\Large\Box$

\tabula{}$\bullet~~$résoudre une équation trigonométrique\dotfill{}$\Large\Box$
\end{coutil}

\titrexo{Chapitre 07 : Nombre dérivé, tangente}\color{titrebleu}\,\hrulefill\,\titrexo{($\rightsquigarrow$ MC)}

\begin{casavoir}
\fontfamily{lmss}\selectfont\small
\textbf{Je sais :}

\tabula{}$\bullet~~$calculer un nombre dérivé par la limite du taux d'accroissement\dotfill{}$\Large\Box$

\tabula{}$\bullet~~$interpréter un nombre dérivé comme pente de tangente\dotfill{}$\Large\Box$

\tabula{}$\bullet~~$lire graphiquement un nombre dérivé\dotfill{}$\Large\Box$

\tabula{}$\bullet~~$déterminer l'équation d'une tangente\dotfill{}$\Large\Box$
\end{casavoir}

\begin{ccalco}
\fontfamily{lmss}\selectfont\small
Je peux, avec ma calculatrice :

\tabula{}$\bullet~~$calculer un nombre dérivé\dotfill{}$\Large\Box$

\tabula{}$\bullet~~$tracer la tangente à une courbe correctement affichée\dotfill{}$\Large\Box$
\end{ccalco}

\titrexo{Chapitre 08 : Fonctions dérivées}\color{titrebleu}\,\hrulefill\,\titrexo{($\rightsquigarrow$ MC)}

\begin{casavoir}
\fontfamily{lmss}\selectfont\small
\textbf{Je sais :}

\tabula{}$\bullet~~$reconnaître la forme générale d'une fonction afin de la dériver\dotfill{}$\Large\Box$

\tabula{}$\bullet~~$utiliser les formules de dérivation pour dériver une fonction\dotfill{}$\Large\Box$

\tabula{}$\bullet~~$utiliser la fonction dérivée afin de déterminer l'équation d'une tangente\dotfill{}$\Large\Box$

\tabula{}$\bullet~~$transformer la dérivée afin d'étudier son signe\dotfill{}$\Large\Box$
\end{casavoir}

\begin{coutil}
\fontfamily{lmss}\selectfont\small
Je peux, avec un logiciel de calcul formel :

\tabula{}$\bullet~~$retrouver la dérivée d'une fonction\dotfill{}$\Large\Box$
\end{coutil}

\titrexo{Chapitre 09 : Suites arithmétiques, géométriques}\color{titrebleu}\,\hrulefill\,\titrexo{($\rightsquigarrow$ MC)}

\begin{casavoir}
\fontfamily{lmss}\selectfont\small
\textbf{Je sais :}

\tabula{}$\bullet~~$reconnaître une suite arithmétique, uns suite géométrique\dotfill{}$\Large\Box$

\tabula{}$\bullet~~$utiliser les formules de récurrence et les formules explicites des suites arithmétiques et géométrique\dotfill{}$\Large\Box$

\tabula{}$\bullet~~$calculer la somme des termes d'une suite arithmétique, d'une suite géométrique\dotfill{}$\Large\Box$

\tabula{}$\bullet~~$démontrer qu'une suite est arithmétique, géométrique\dotfill{}$\Large\Box$

\tabula{}$\bullet~~$étudier le sens de variation d'une suite arithmétique, d'une suite géométrique\dotfill{}$\Large\Box$
\end{casavoir}

\titrexo{Chapitre 10 : Calcul vectoriel, produit scalaire}\color{titrebleu}\,\hrulefill

\begin{casavoir}
\fontfamily{lmss}\selectfont\small
\textbf{Je sais :}

\tabula{}$\bullet~~$calculer un produit scalaire à l'aide de l'une des trois formes\dotfill{}$\Large\Box$

\tabula{}$\bullet~~$étudier l'orthogonalité de deux vecteurs\dotfill{}$\Large\Box$

\tabula{}$\bullet~~$utiliser un produit scalaire pour déterminer un angle\dotfill{}$\Large\Box$
\end{casavoir}

\begin{ccalco}
\fontfamily{lmss}\selectfont\small
Je peux, avec ma calculatrice :

\tabula{}$\bullet~~$déterminer un angle connaissant son cosinus\dotfill{}$\Large\Box$
\end{ccalco}

\pagebreak

\titrexo{Chapitre 11 : Applications de la dérivation}\color{titrebleu}\,\hrulefill\,\titrexo{($\rightsquigarrow$ MC)}

\begin{casavoir}
\fontfamily{lmss}\selectfont\small
\textbf{Je sais :}

\tabula{}$\bullet~~$étudier les variations d'une fonction et dresser son tableau de variations\dotfill{}$\Large\Box$

\tabula{}$\bullet~~$déterminer maximum/minimum\dotfill{}$\Large\Box$

\tabula{}$\bullet~~$faire le lien entre extremum et dérivée\dotfill{}$\Large\Box$
\end{casavoir}

\begin{ccalco}
\fontfamily{lmss}\selectfont\small
Je peux, avec ma calculatrice :

\tabula{}$\bullet~~$tracer la courbe d'une fonction dans une fenêtre adaptée\dotfill{}$\Large\Box$

\tabula{}$\bullet~~$utiliser les outils graphiques de la calculatrice (max, min, racine, etc)\dotfill{}$\Large\Box$
\end{ccalco}

\titrexo{Chapitre 12 : Variables aléatoires}\color{titrebleu}\,\hrulefill\,\titrexo{($\rightsquigarrow$ MC)}

\begin{casavoir}
\fontfamily{lmss}\selectfont\small
\textbf{Je sais :}

\tabula{}$\bullet~~$déterminer les valeurs prises par une variable aléatoire\dotfill{}$\Large\Box$

\tabula{}$\bullet~~$déterminer la loi de probabilité d'une variable aléatoire\dotfill{}$\Large\Box$

\tabula{}$\bullet~~$calculer l'espérance, la variance et l'écart-type d'une variable aléatoire\dotfill{}$\Large\Box$

\tabula{}$\bullet~~$interpréter l'espérance d'une variable aléatoire\dotfill{}$\Large\Box$

\tabula{}$\bullet~~$déterminer si un jeu est équitable, favorable au joueur, etc\dotfill{}$\Large\Box$
\end{casavoir}

\titrexo{Chapitre 13 : Fonction exponentielle}\color{titrebleu}\,\hrulefill\,\titrexo{($\rightsquigarrow$ MC)}

\begin{casavoir}
\fontfamily{lmss}\selectfont\small
\textbf{Je sais :}

\tabula{}$\bullet~~$exploiter le fait qu'une exponentielle est strictement positive\dotfill{}$\Large\Box$

\tabula{}$\bullet~~$manipuler les propriétés algébrique de l'exponentielle (relation fonctionnelle, etc)\dotfill{}$\Large\Box$

\tabula{}$\bullet~~$résoudre des (in)équations avec de l'exponentielle\dotfill{}$\Large\Box$

\tabula{}$\bullet~~$étudier le signe d'une expression faisant intervenir de l'exponentielle\dotfill{}$\Large\Box$

\tabula{}$\bullet~~$représenter l'allure de la courbe des fonctions $\mathsf{x \mapsto e^{kx}}$\dotfill{}$\Large\Box$

\tabula{}$\bullet~~$dériver des fonctions faisant intervenir de l'exponentielle\dotfill{}$\Large\Box$

\tabula{}$\bullet~~$interpréter une évolution exponentielle (lien avec les suites géométriques)\dotfill{}$\Large\Box$

\end{casavoir}

\titrexo{Chapitre 14 : Équations cartésiennes, droites et cercles}\color{titrebleu}\,\hrulefill

\begin{casavoir}
\fontfamily{lmss}\selectfont\small
\textbf{Je sais :}

\tabula{}$\bullet~~$utiliser les formules d'Al-Kashi\dotfill{}$\Large\Box$

\tabula{}$\bullet~~$manipuler une équation cartésienne de droite (vecteur directeur, vecteur normal, etc)\dotfill{}$\Large\Box$

\tabula{}$\bullet~~$manipuler une équation cartésienne de cercle (centre/rayon, utilisation de la forme canonique, etc)\dotfill{}$\Large\Box$
\end{casavoir}

\vfill{}

\begin{center}
	\includegraphics[scale=1]{asuivre}
\end{center}

\end{document}