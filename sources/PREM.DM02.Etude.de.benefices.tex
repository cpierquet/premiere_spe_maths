% !TeX TXS-program:compile = txs:///lualatex

\documentclass[a4paper,11pt]{article}
\usepackage[]{cp-base} %avec options possibles parmi breakable (tcbox), sujetl (exos),  (pour faire "comme avant"), etc...
\graphicspath{{./graphics/}}
%variables
\donnees[%
	classe=1\up{ère} 2M2,
	matiere={[SPÉ.MATHS]},
	typedoc=DM,
	numdoc=02,
	mois=Septembre,
	annee=2021,
	titre={Étude de bénéfices}
	]
	
%formatage
\author{Pierquet}
\title{\nomfichier}
\hypersetup{pdfauthor={Pierquet},pdftitle={\nomfichier},allbordercolors=white,pdfborder=0 0 0,pdfstartview=FitH}
%divers
\lhead{\entete{\matiere}}
\chead{\entete{\lycee}}
\rhead{\entete{\classe{} - \mois{} \annee}}
%\rhead{\entete{\classe{} - Chapitre }}
\lfoot{\pied{\matiere}}
\cfoot{\logolycee{}}
\rfoot{\pied{\numeropagetot}}
\fancypagestyle{entetedm}{\fancyhead[L]{\entete{\matiere{} À rendre avant le\ldots}}}

\begin{document}

\pagestyle{fancy}

\thispagestyle{entetedm}

\part{DM02 - Étude de bénéfices}

\medskip

Une entreprise fabrique des petits meubles. Pour des raisons de stockage, la production mensuelle $x$ est comprise entre 0 et 800 unités.

Le coût total de fabrication mensuel, exprimé en euros, est donné par la fonction $C$ définie sur $\intervFF{0}{800}$ par :\[ C(x)=0,05x^2+20x+1\,250. \]
On a tracé ci-dessous la courbe de la fonction $C$ ainsi que la droite $\mathcallig{R}$ représentant la fonction recette, le tout dans un repère orthogonal.

\begin{center}
	\tunits{0.18}{0.18}
	\tdefgrille{0}{85}{5}{5}{0}{50}{50}{5}
	\begin{tikzpicture}[x=\xunit cm,y=\yunit cm]
		%AXES & GRILLES
		\tgrilles[orange!50,line width=0.4pt,densely dashed] 
		\axestikz
		\foreach \x in {0,5,...,80}
			\FPeval{calx}{clip(\x*10)}
			\draw[line width=1.25pt] (\x,4pt) -- (\x,-4pt) node[below] {\num{\calx}} ;
		%\axextikz{0,50,...,800} ;
		\foreach \y in {0,5,...,45}
			\FPeval{caly}{clip(\y*1000)}
			\draw[line width=1.25pt] (4pt,\y) -- (-4pt,\y) node[left] {\num{\caly}} ;
		%COURBE
		\draw[line width=1.25pt,red,domain=0:80] plot (\x,{0.005*\x*\x+0.20*\x+1.250}) ;
		\draw[line width=1.25pt,blue,domain=0:80] plot (\x,{0.475*\x}) ;
		\draw[darkgray,fill=darkgray] (65,30.875) circle(3pt) ;
		\draw (65,30.875) node[below right] {$(650\,;\,30\,875)$} ;
	\end{tikzpicture}
\end{center}

\begin{enumerate}
	\item À l’aide du graphique, déterminer le prix de vente exact de chaque meuble.
	\item Calculer la recette, puis le bénéfice correspondant à 150 meubles fabriqués et vendus.
	\item Graphiquement, sur quel intervalle l’entreprise réalise-t-elle des bénéfices ? Justifier.
	\item 
	\begin{enumerate}
		\item Donner la recette mensuelle $R(x)$, exprimée en euros, correspondant à $x$ meubles vendus.
		\item En déduire le bénéfice mensuel $B(x)$, exprimé en euros, correspondant à $x$ meubles fabriqués et vendus.
	\end{enumerate}
	\item
	\begin{enumerate}
		\item Déterminer, en détaillant, la forme canonique de la fonction $B$.
		\item Déterminer, en justifiant, le nombre de meubles à fabriquer et à vendre pour que le bénéfice soit maximal. Préciser la valeur de ce bénéfice maximal.
	\end{enumerate}
	\item 
	\begin{enumerate}
		\item Résoudre l'équation $B(x)=0$.
		\item Déterminer la forme factorisée de $B(x)$.
		
		\textbf{NB} : On pourrait ainsi retrouver, par calculs, le résultat de la question \ptno{3} !
	\end{enumerate}
	\item Déterminer, par la méthode de votre choix, les quantités de meubles à fabriquer et à vendre pour que le bénéfice soit de :
	\begin{enumerate}
		\item $-1\,250$\,€ ;
		\item 1\,000\,€.
	\end{enumerate}
\end{enumerate}

\end{document}