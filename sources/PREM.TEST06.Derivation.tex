% !TeX TXS-program:compile = txs:///arara
% arara: lualatex: {shell: no, synctex: yes, interaction: batchmode}
% arara: pythontex: {rerun: modified} if found('pytxcode', 'PYTHONTEX#py')
% arara: lualatex: {shell: no, synctex: yes, interaction: batchmode} if found('pytxcode', 'PYTHONTEX#py')
% arara: lualatex: {shell: no, synctex: yes, interaction: batchmode} if found('log', '(undefined references|Please rerun|Rerun to get)')

\documentclass[a4paper,11pt]{article}
\usepackage[revgoku]{cp-base}
\graphicspath{{./graphics/}}
%variables.
\donnees[classe={1\up{ère} 2M2},matiere={[SPÉ.MATHS]},mois={Mardi 1\up{er} Mars},annee=2022,typedoc=TEST~,numdoc=6,duree={15 minutes}]
%formatage
\author{Pierquet}
\title{\nomfichier}
\hypersetup{pdfauthor={Pierquet},pdftitle={\nomfichier},allbordercolors=white,pdfborder=0 0 0,pdfstartview=FitH}
%divers
\lhead{\entete{Durée : \duree}}
\chead{\entete{\lycee}}
\rhead{\entete{\classe{} - \mois{} \annee}}
\lfoot{\pied{\matiere}}
\cfoot{\logolycee{}}
\fancypagestyle{sujetA}{\fancyhead[R]{\entete{\classe{}A - \mois{} \annee}}}
\fancypagestyle{sujetB}{\fancyhead[R]{\entete{\classe{}B - \mois{} \annee}}}

\begin{document}

\pagestyle{fancy}

\part{TEST06 - Dérivation}%SUJETA

\setcounter{numexos}{0}

\medskip

\nomprenomtcbox

\medskip

\exonum{}

\begin{enumerate}
	\item On considère la fonction $f$ définie sur $\R^+$ par $f(x)=\dfrac{\sqrt{x}}{x^2+1}$.
	
	À l'aide de la calculatrice, donner une valeur de $f'(1)$.
	\item On considère une fonction $g$ définie sur $\intervFF{-5}{4}$ dont la courbe représentative $\mathscr{C}_g$ est donnée ci-dessous.
	
	Les tangentes aux points d'abscisses $-2$ et $3$ sont tracées.
	\begin{center}
		\begin{tikzpicture}[xmin=-5,xmax=4,ymin=-3,ymax=3,x=0.75cm,y=0.75cm]
			\tgrillep \axestikz* \axextikz*{-5,-4,...,3} \axeytikz*{-3,-2,...,2}
			\draw (1,-4pt) node[below,font=\small\sffamily] {1} ;
			\draw (-4pt,1) node[left,font=\small\sffamily] {1} ;
			\draw (-2pt,-2pt) node[below left,font=\small\sffamily] {0} ;
			\def\liste{-5/-1/2§-2/2.25/0§1/-2.5/0§3/0/2§4/3/10}
			\splinetikz[affpoints=false,liste=\liste,coeffs=3§3§3/2§3]
			\filldraw[blue] (-2,2.25) circle[radius=1.75pt] ;
			\draw[line width=1.5pt,blue] (-5,2.25)--(4,2.25) ;
			\filldraw[ForestGreen] (3,0) circle[radius=1.75pt] ;
			\draw[line width=1.5pt,ForestGreen] (1.5,-3)--(4,2) ;
		\end{tikzpicture}
	\end{center}
	Par lecture graphique, déterminer la valeur de :
	\begin{enumerate}
		\item $g'(-2)$ ;
		\item $g'(3)$.
	\end{enumerate}
\end{enumerate}

\papierseyes*{11}{2}~~\papierseyes*{11}{2}

\bigskip

\exonum{}

\begin{enumerate}
	\item On considère la fonction $f$ définie sur $\R$ par $f(x)=x^4$. On note $\mathscr{C}_f$ sa courbe représentative.
	\begin{enumerate}
		\item Déterminer la dérivée $f'$ de la fonction $f$.
		\item Calculer $f(1)$ et $f'(1)$.
		\item Déterminer une équation de $T_1$, tangente à $\mathscr{C}_f$ au point d'abscisse 1.
	\end{enumerate}
	\item Déterminer la dérivée des fonctions suivantes :
	\begin{enumerate}
		\item $f(x)=x+\dfrac{1}{x}$ ;
		\item $g(x)=4x^3-12x^2$.
	\end{enumerate}
\end{enumerate}

\papierseyes*{11}{5}~~\papierseyes*{11}{5}

\end{document}